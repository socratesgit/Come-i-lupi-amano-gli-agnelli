\documentclass[12pt]{book}

\usepackage{verse}
\PassOptionsToPackage{bookmarks, colorlinks=false, hidelinks}{hyperref}
% Use PoetryTeX; http://www.ctan.org/pkg/poetrytex
\usepackage[numberpoems, clearpageafterpoem, useincipits]{poetrytex}
\usepackage{dirtytalk}
\usepackage[T1]{fontenc}
\usepackage[utf8]{inputenc}
\usepackage[italian]{babel}

% Use the PA5 paper size
\usepackage[paperwidth=140mm,paperheight=210mm]{geometry}
\def\changemargin#1#2{\list{}{\rightmargin#2\leftmargin#1}\item[]}
\let\endchangemargin=\endlist 

\renewcommand{\pttitle}{Come i lupi amano gli agnelli}
\renewcommand{\ptsubtitle}{}
\renewcommand{\ptauthor}{Ivan Masnari}
\renewcommand{\ptdate}{2020/01/07}
\renewcommand{\ptdedication}{
Siedi più a me vicino, \\
guardami con occhi allegri: \\
ecco il quaderno azzurro \\
dei miei versi infantili.
}


\begin{document}


\maketitle
\makededication


% Number pages with small roman numerals (i, ii, iii, iv...)
\frontmatter


% TOP %
\renewcommand*{\topname}{Indice} % Name for the table of poems
\maketop


\section{Introduzione}
Chi sono? \\*
Con le mie ragioni, sono i libri che ho letto. \\*
Allo stesso modo, per contrasto, \\*
con le mie incomprensioni e le mie partigianerie, \\*
sono i libri che non ho letto. \\*
Gramsci diceva \\*
\say{ognuno è conformista di un qualche conformismo}. \\*
La mia presunta originalità non fa eccezione. \\*
Se non cito tutte le fonti \\*
è perchè il plagio è diventato, per me, \\*
una seconda natura: \\*
rubo senza accorgermene, ladro inconsapevole. \\*
Tuttavia, se il tutto è più della somma delle sue parti, \\*
riconosco la mia impronta nell'inedito ordine \\*
in cui riesco a comporre le vecchie parole. \\*
Chi sono? Io - rispondo. \\*
L'abitudine grammaticale \\*
impone l'uso del pronome di prima persona, \\*
sostenuta, in questo, dalla nostra coazione a ripetere \\*
a noi stessi che qualcosa \\*
si deve pur essere. \\*
Alcuni ritengono di doversi tributare molta importanza,\\*
e allegano meriti e raccomandazioni. \\*
Altri dichiarano a gran voce la loro appartenenza \\*
ad un credo, una fazione politica. \\*
\newpage
Per me, la certezza di esistere \\*
non si declina in senso identitario. \\*
Sono cosciente che, se sono quello che sono,\\*
lo sono per caso. \\*
Avrei potuto non essere amato, \\*
e non ne avrei capito il valore. \\*
Il cibo mi poteva mancare, \\*
e forse ne sprecherei meno. \\*
Questa realizzazione mi permette di non irrigidirmi \\*
nella mia prospettiva, per forza di cose, situata. \\*
A volte, anzi, mi guardo con un certo scetticismo, \\*
come si guarda un estraneo un po'matto. \\*
Riconosco che quelle che chiamo evidenze \\*
sono tali più in forza della mia convinzione testarda, \\*
che per una loro intrinseca intellegibilità. \\*
Riconosco, parimenti, la necessità di avere delle certezze \\*
per esercitare il dubbio in modo sano. \\*
Sono persuaso che ciò che credo vero faccia di me \\*
la persona che sono. \\*
Ma cambio spesso idea. \\*


% Start numbering pages with normal arabic numerals.
\mainmatter

\begin{poem}{}{}

\settowidth{\versewidth}{non meno è ciò che meriti.}

\begin{altverse}
% Incipits are used in the ToP if no title is given.
\incipit{Vedo oltre a te, un poco oltre}:\\*
non più ciò che sei, \\*
non meno è ciò che meriti. \\*
Amando oltre a te, un poco oltre, \\*
amo il tuo riflesso \\*
e non per questo amo meno: \\*
amo te, un poco oltre. \\*
\end{altverse}

\end{poem}

\begin{poem}{}{}

\settowidth{\versewidth}{non meno è ciò che meriti.}

\begin{altverse}
% Incipits are used in the ToP if no title is given.
\incipit{Questa, mia cara, è la poesia}.\\*
Questa è la strofa che ne è il corpo \\*
questo il verso che ne è il braccio, \\*
la gamba, la voce e tutto il resto: \\*
sillabico per lunga tradizione \\*
si sceglie poi il prefisso per l'occasione, \\*
che sia ende, dodecasillabo, \\*
oppure un breve quinario o un trisillabo \\*
brevissimo. \\* 
\end{altverse}

\begin{altverse}
Qui sono le rime baciate o invertite, \\*
pervertite o sdrucciole e le assonanze. \\*
E qui, invece, qualche perifrasi, \\*
anafore, epifore, qualche trucco, sai, \\*
per rimescolare le solite quattro parole. \\*
\end{altverse}

\begin{altverse}
Qui, vedi amore, è un'allusione a certi simbolisti, \\*
e qua c'è un verso plagiato a Verlain, \\*
questo a Blok e questo.. \\*
no, in verità, questo sarebbe mio.. \\* 
\end{altverse}

. . . . . . . . .

\begin{altverse}
Mentre ti spiego, tu \\*
confusa mi guardi, \\*
giri un poco gli occhi, inclini \\*
la testa da un lato, mi chiedi: \\*
\say{Ma il poeta? \\*
Il poeta dov'è?} \\*
\end{altverse}

\begin{altverse}
Ma il poeta è dietro a tutto! \\*
dietro alle quinte \\*
come un buon regista alla sua prima. \\*
\end{altverse}

\begin{altverse}
E mi chiedi: \\*
\say{E io? \\*
Io dove sono?} \\*
\end{altverse}

\begin{altverse}
Ma tu sei qui, di fianco a me, \\*
a leggere la poesia che, per te, ho scritta. \\*
\end{altverse}


\end{poem}

\begin{poem}{Routine}{}

\settowidth{\versewidth}{non meno è ciò che meriti.}

\begin{altverse}
% Incipits are used in the ToP if no title is given.
\incipit{I miei giorni girano come aghi ubriachi}:\\*
al contatto della mia pelle \\*
smussano il loro filo. \\*
\end{altverse}


\begin{altverse}
La cruna sformata \\*
fa passare più d'una carovana. \\*
\end{altverse}


\end{poem}

\begin{poem}{Sonetto}{}

\settowidth{\versewidth}{ha disatteso tutte le promesse e le belle speranze:}
\begin{changemargin}{-1cm}{-2cm} 
\begin{altverse}
% Incipits are used in the ToP if no title is given.
\incipit{\quad \qquad Acquario, il prodigo dispensiere celeste},\\*
ha disatteso tutte le promesse e le belle speranze: \\*
non ha allungato su di noi la sua mano, \\*
non ha esaudito le nostre preghiere. 
\end{altverse}

\begin{altverse}
Le nostre bocche sono vuote, \\*
le nostre parole arida sabbia che asseta \\*
le nostre anime come uadi avidi \\*
d'acqua o d'amore. 
\end{altverse}

\begin{altverse}
Abbiamo corpi ma non c'è calore \\*
nel loro contatto, nè colore \\*
nei baci caduchi, esangui, \\*
simili alle foglie ingiallite e moriture. 
\end{altverse}

\begin{altverse}
Solo, ora, ci consola il pallido desiderio \\*
di un'assenza languida come il tramonto. \\*
\end{altverse}
\end{changemargin}
\end{poem}

\begin{poem}{}{}

\settowidth{\versewidth}{mentre la tua nostalgia permane nei laghi  }

\begin{altverse}
% Incipits are used in the ToP if no title is given.
\incipit{È la tua assenza, ora, che s'ingruma}\\*
nella forma incerta d'essenza: \\*
nulla che s'annulla  \\*
in un vuoto su misura, \\*
nel buco adatto ad ospitare \\*
il tuo ricordo. 
\end{altverse}

\begin{altverse}
È questa la presenza che mi ferisce \\*
con la sua stanca indifferenza. 
\end{altverse}


\begin{altverse}
Non posso trattenerti, eppure, \\*
non posso lasciarti fuggire.\\*
\end{altverse}


\end{poem}

\begin{poem}{}{}

\settowidth{\versewidth}{mentre la tua nostalgia permane nei laghi  }

\begin{altverse}
% Incipits are used in the ToP if no title is given.
\incipit{Guardarti è come} \\*
origliare a una porta chiusa. 
\end{altverse}

\begin{altverse}
Le mie parole risalgono \\*
la corrente muta, \\*
le vene silenziose, \\*
e là si perdono. 
\end{altverse}

\begin{altverse}
Io non ti conosco, \\*
non so \\*
chi sei. 
\end{altverse}

\begin{altverse}
Oltre l'inopportuna apparenza \\*
e la coltre che la vela, \\*
tu devi essere non più spessa \\*
d'un filo. 
\end{altverse}


\end{poem}

\begin{poem}{Plazer}{}

\settowidth{\versewidth}{non meno è ciò che meriti.}

\begin{altverse}
% Incipits are used in the ToP if no title is given.
\incipit{Amo l'apparenza delle cose fragili},\\*
la loro sostanza impermanente: \\*
amo quel niente che sono.
\end{altverse}

\begin{altverse}
Amo i calici di cristallo, \\*
l'aria piena di sole, il sorriso \\*
appena disegnato sulle labbra \\*
a cui ho sorriso.
\end{altverse}

\begin{altverse}
Amo le cose fragili, \\*
quelle non facili da conservare, \\*
quelle che ti capita di dimenticare. \\*
\end{altverse}

\end{poem}

\begin{poem}{Epigramma I}{}

\settowidth{\versewidth}{non meno è ciò che meriti.}

\begin{altverse}
% Incipits are used in the ToP if no title is given.
\incipit{I vent'anni minacciano da vicino},\\*
inaspettati, perchè il tempo \\*
si misura solo nelle scadenze.
\end{altverse}

\begin{altverse}
Disperato o più stanco, \\*
mi sorprendo a vivere ancora. 
\end{altverse}

\begin{altverse}
Ho imparato che la rinuncia \\*
non è una scelta.
\end{altverse}

\end{poem}

\begin{poem}{Epigramma II}{}

\settowidth{\versewidth}{non meno è ciò che meriti.}

\begin{altverse}
% Incipits are used in the ToP if no title is given.
\incipit{Amburgo è sommersa}:\\*
le strade sono acquitrini.
\end{altverse}

\begin{altverse}
L'alluvione ha riempito \\*
le vie di fango.
\end{altverse}

\begin{altverse}
Il mondo cancellato \\*
è un'enorme palude. 
\end{altverse}

\begin{altverse}
Uomini, come rane, \\*
gracidano in ogni piazza.
\end{altverse}

\begin{altverse}
Qui, cammino solo e sono \\*
ciò che resta di me: \\*
poco, ma è quel basta.

\end{altverse}

\end{poem}

\begin{poem}{Epigramma III}{}

\settowidth{\versewidth}{non meno è ciò che meriti.}

\begin{altverse}
% Incipits are used in the ToP if no title is given.
\incipit{Scendo nell'ipogeo della città},\\*
perso nell'anonimo.
\end{altverse}

\begin{altverse}
Scivolo non visto, \\*
pieno di nulla.
\end{altverse}

\begin{altverse}
Ho freddo, mi stringo \\*
più stretto ai miei libri.
\end{altverse}

\begin{altverse}
Il mendicante allunga una mano. \\*
Io non gli do un soldo.
\end{altverse}

\end{poem}

\begin{poem}{Epigramma IV}{}

\settowidth{\versewidth}{non meno è ciò che meriti.}

\begin{altverse}
% Incipits are used in the ToP if no title is given.
\incipit{Sono secoli che nessuno scrive in esametri}:\\*
l'epica è morta d'inedia.
\end{altverse}

\begin{altverse}
Ora, Omero si legge \\*
in pratiche edizioni tascabili.
\end{altverse}

\begin{altverse}
E' un fatto di costituzione. \\*
Oggigiorno i poeti sono troppo fragili.
\end{altverse}

\begin{altverse}
Il loro corpo troppo sottile \\*
è quello d'una gru.
\end{altverse}

\begin{altverse}
Non hanno più i polmoni per l'epos. \\*
A malapena sembrano stare in piedi.
\end{altverse}

\end{poem}

\begin{poem}{}{}

\settowidth{\versewidth}{non meno è ciò che meriti.}

\begin{changemargin}{-1cm}{-3cm} 

\begin{altverse}
% Incipits are used in the ToP if no title is given.
\quad \qquad \incipit{ Il silenzio abita le stanze buie},\\*
le case svuotate di recente per il trasloco.\\*
Puoi sentirlo quando sali, \\*
senza rumore, su una scala, \\*
e dopo la breve ascensione, \\*
ti ritrovi in soffitta. \\*
Sembra la Morte - una sua fedele immagine -\\*
Forse è soltanto l’estensione propria della tua vita, \\*
ma la retorica ti ha preso la mano \\*
e lasciandoti trascinare da una facile metafora, dici: \\*
“Questa è la fine”.
\end{altverse}


\end{changemargin}


\end{poem}

\begin{poem}{}{}

\settowidth{\versewidth}{non meno è ciò che meriti.}

\begin{altverse}
% Incipits are used in the ToP if no title is given.
\incipit{Lascio le pianure dorate}.\\*
Le forme del sole, già sbiadite, \\*
si perdono dietro alla montagna scura. \\*
Quanto del mondo sopravvive nell'ombra?\\*
La foglia e la mano che la coglieva \\*
sono due punti inestesi, \\*
ambigui nella loro vicinanza: \\*
si confondono, ora, \\*
coincidono.
\end{altverse}

\end{poem}

\begin{poem}{Dachau}{31/12/2014}

\settowidth{\versewidth}{non meno è ciò che meriti.}

\begin{changemargin}{-2cm}{-3cm} 

\begin{altverse}
% Incipits are used in the ToP if no title is given.
\quad \qquad \incipit{Ecco l'angelo nero nella sua dolorosa grazia},\\*
ecco l'angelo vestito di sangue e lacrime, dire: \\*
"Qui vive la sofferenza senza nome,\\*
la morte e la memoria dell'orrore. \\*
Vedi immagini d'uomini percorrere il perimetro del campo,\\*
ma troppo lieve è il loro passo per lasciare un'orma,\\*
troppo debole la loro voce\\*
per bucare il silenzio.\\*
Tu ricorda per loro, come puoi, come devi.\\*
Tu ricorda agli uomini\\*
che il peggio non ha fine."
\end{altverse}


\end{changemargin}

\end{poem}

\begin{poem}{Elegia}{}
\begin{changemargin}{-1cm}{-2cm}
\settowidth{\versewidth}{non meno è ciò che meriti.}

\begin{altverse}
% Incipits are used in the ToP if no title is given.
\incipit{\quad \qquad Io sono un ragazzo triste},\\*
con una faccia triste\\*
che fa cose tristi e inutili.\\*
Ho due braccia, due gambe tristi\\*
ho due tondi occhi tristi.
\end{altverse}

\begin{altverse}
La tristezza ama me\\*
e io amo le cose tristi e morte:\\*
io amo la mia tristezza\\*
e l'accarezzo\\*
come si fa con un gatto nero,\\*
e l'accarezzo\\*
come si fa con un cane bianco.\\*
E per questo io amo la mia casa,\\*
perchè è un luogo di dolore\\*
e lì la mia tristezza ci sta calda\\*
e amo la mia finestra\\*
perchè ha delle sbarre\\*
e sembra quasi una prigione\\*
e amo l'uccello ferito,\\*
fuori nel mio giardino:\\*
ha un'ala spezzata e non può volare.
\end{altverse}

\begin{altverse}
Tutti gli altri mi dicono:\\*
\say{La vita è bella!\\*
La vita è lunga!}\\*
E io abbozzo e svicolo\\*
e provo a ridere per compiacerli,\\*
ma sono un ragazzo triste\\*
e il sorriso mi si gela nei denti\\*
e le parole felici mi si incastrano in gola.
\end{altverse}

\begin{altverse}
Quelli mi dicono:\\*
\say{La vita fa i limoni,\\*
tu facci una limonata!}\\*
Ma io sono un ragazzo triste\\*
e penso che è una frase fatta\\*
e penso che è un falso proverbio.
\end{altverse}


\begin{altverse}
Allora quelli mi dicono:\\*
\say{Vattene! ci metti tristezza.}\\*
E io dico \say{volentieri}\\*
ma rimango,\\*
\say{ancora qualche minuto} dico\\*
e mi invento qualche grave lutto,\\*
un incidente, una donna\\*
e quelli mi offrono una birra\\*
e parlano tra loro e sono tutti ragazzi felici\\*
e allora anch'io sono un ragazzo triste e felice.
\end{altverse}
\end{changemargin}
\end{poem}

\begin{poem}{Le Ciel et la Terre}{incipit}

\settowidth{\versewidth}{non meno è ciò che meriti.}
\begin{changemargin}{-1cm}{-2cm} 
\begin{altverse}
% Incipits are used in the ToP if no title is given.
\incipit{}
\textit{\quad \qquad Quando verrai, aspetterò al porto. \\*
Avrò la camicia azzurra, quella elegante. \\*
Non puoi sbagliare.}
\end{altverse}

\begin{altverse}
\textit{Questa è una piccola isola, malservita:\\*
solo un porto, solo una nave, \\*
solo un uomo in azzurro, là sul molo, che saluta.}
\end{altverse}
\end{changemargin}
\end{poem}


\begin{poem}{I}{}

\settowidth{\versewidth}{non meno è ciò che meriti.}
\begin{changemargin}{-1cm}{-2cm} 
\begin{altverse}
% Incipits are used in the ToP if no title is given.
\quad \qquad \incipit{Ero io “l’uomo dell’altro secolo”},\\*
in ritardo su tutto.
\end{altverse}

\begin{altverse}
Tu eri il pieno, il centro, e io \\*
il vuoto che circonda.
\end{altverse}

\begin{altverse}
Le percentuali ci davano perdenti: \\*
abbiamo sconfessato le statistiche;
\end{altverse}

\begin{altverse}
non ci ha uccisi la città dolente, \\*
né la stanca indifferenza della gente.
\end{altverse}

\begin{altverse}
Ti ripeto che niente può accadere, \\*
che la felicità è il frutto di infinite sottrazioni,
\end{altverse}

\begin{altverse}
che domani sorgerà lo stesso sole, \\*
che questa persistenza è il nostro destino.
\end{altverse}
\end{changemargin}
\end{poem}

\begin{poem}{II}{}

\settowidth{\versewidth}{non meno è ciò che meriti.}

\begin{altverse}
% Incipits are used in the ToP if no title is given.
\incipit{“No, non c’è sopravvivenza.}.\\*
Non credo nella vita dopo la morte.”
\end{altverse}

\begin{altverse}
“Ma nell’amore, \\*
nell’amore prima della morte?
\end{altverse}

\begin{altverse}
In quello credi?” \\*
“Sì, in quello credo.”
\end{altverse}

\end{poem}

\begin{poem}{III}{}

\settowidth{\versewidth}{non meno è ciò che meriti.}
\begin{changemargin}{-1cm}{-2cm} 
\begin{altverse}
% Incipits are used in the ToP if no title is given.
\quad \qquad \incipit{Parlo con il cuore nella mano},\\*
la mano spinta contro il tuo petto.
\end{altverse}


\begin{altverse}
Tu mi guardi, osservi che un Agosto\\*
non poteva essere più mite.
\end{altverse}


\begin{altverse}
La notte non ci contraddice,\\*
nessuno, per farlo, busserà alla porta.
\end{altverse}


\begin{altverse}
Le nostre parole non sono fatte per durare,\\*
non sopravvivono per ripetersi e perseguitarci.
\end{altverse}


\begin{altverse}
Le nostre voci, nel buio,\\*
mormorano cose gentili.
\end{altverse}
\end{changemargin}
\end{poem}

\begin{poem}{Paraclausithyron}{}

\settowidth{\versewidth}{non meno è ciò che meriti.}
\begin{changemargin}{-2cm}{-3cm} 
\begin{altverse}
% Incipits are used in the ToP if no title is given.
\incipit{\quad \qquad Un’altra poesia scrivo, che non leggerai}: \\*
lo spazio che ci divide, ti nasconde alla mia voce. \\*
È sottointesa la sterile pretesa che il tempo sia reversibile.
\end{altverse}

\begin{altverse}
Parlo all’immagine muta – il mio ricordo di te. \\*
“Avrei preferito evitare” le dico \\*
“non volevo finisse in questo modo”.
\end{altverse}

\begin{altverse}
È ridicolo sperare che tu possa rispondermi. \\*
È ridicolo anche solo pensarlo. \\*
Sono io quello che batte a una porta chiusa.
\end{altverse}

\end{changemargin}
\end{poem}

\begin{poem}{}{}

\settowidth{\versewidth}{non meno è ciò che meriti.}


\begin{altverse}
% Incipits are used in the ToP if no title is given.
\incipit{Se ti indicassi }\\*
avrei perso la mia saggezza.
\end{altverse}

\begin{altverse}
Barattando la forma \\*
per il contenitore, \\*
la mente non coglie che spoglie.
\end{altverse}


\end{poem}

\begin{poem}{Melanconia}{}

\settowidth{\versewidth}{non meno è ciò che meriti.}


\begin{altverse}
% Incipits are used in the ToP if no title is given.
\incipit{La Mestizia striscia sulla mia spalla }\\*
come una vipera derelitta\\*
come una vecchia amica malata.
\end{altverse}

\begin{altverse}
\say{Che fai?} mi chiede.\\*
\say{Ti evito} rispondo.
\end{altverse}

\begin{altverse}
Ma lei sa che mento,\\*
e lei sa che scrivo meglio\\*
con lei sulla spalla,\\*
come una vecchia derelitta\\*
come una malata amica vipera,\\*
ma sincera.
\end{altverse}


\end{poem}

\begin{poem}{Verwandlung}{}

\settowidth{\versewidth}{non meno è ciò che meriti.}


\begin{altverse}
% Incipits are used in the ToP if no title is given.
\incipit{Ho sentito notti farsi poesia }\\*
d'un tratto inaspettate, come \\*
i gesti farsi danze insospettate \\*
di ritmi minimali solo nel suono. 
\end{altverse}

\begin{altverse}
Ho notato tratti confusi e macchie\\*
di colore mai veduti su volti\\*
nemmeno visti per intero\\*
mescolarsi nelle prese\\*
di coraggio e posizioni\\*
e scolorare poi nelle rinunce.
\end{altverse}

\begin{altverse}
Il vortice li prese e mi ritrovai\\*
in altro tempo luogo forma\\*
e le leggi si invertivano e dovetti\\*
rinascere sotto mentite spoglie\\*
o sotto nessuna.
\end{altverse}
\end{poem}

\begin{poem}{Milano}{}


\begin{altverse}

Questa notte sognavo\\*
una chiatta ormeggiata\\*
davanti al Duomo.\\*
Mentre sull'acqua\\*
veleggiavano in cerchio\\*
due gabbiani,\\*
nella piazza di marmo,\\*
abbacinata, una paranca\\*
estraeva sabbia fine\\*
dalla cavità di cui le chiatte,\\*
con buona approssimazione,\\*
sono interamente costituite.
\end{altverse}

\begin{altverse}
Reminiscenze queste,\\*
dei racconti del mio vecchio\\*
che ricorda un tempo\\*
in cui Milano,\\*
coperta di canali,\\*
era una città fluviale.\\*
Milano,\\*
piccola Pietroburgo\\*
adagiata sulla pianura,\\*
aveva vie d'acqua \\*
dove i suoi palazzi,\\*
raddoppiati nel riflesso,\\*
si specchiavano.
\end{altverse}

\begin{altverse}
Ora i canali\\*
sono stati interrati,\\*
il letto degli antichi fiumi\\*
ospita il metrò\\*
e i gabbiani \\*
che abitano i miei sogni\\*
sopravvivono solo\\*
nelle parole dei vecchi,\\*
piene di uno stupore\\*
ormai quasi dimenticato,\\*
per un miracolo tanto semplice\\*
come una chiatta\\*
piena di sabbia\\*
ormeggiata davanti al Duomo.\\*
\end{altverse}


\end{poem}

\begin{poem}{Haiku I}{}

\settowidth{\versewidth}{non meno è ciò che meriti.}

\begin{altverse}
% Incipits are used in the ToP if no title is given.
\incipit{Non più che neve:}\\*
candido germoglio,\\*
viso d'inverno.
\end{altverse}

\end{poem}

\begin{poem}{Haiku II}{}

\settowidth{\versewidth}{non meno è ciò che meriti.}

\begin{altverse}
% Incipits are used in the ToP if no title is given.
\incipit{Pallida brina,}\\*
soffia il gelo come\\*
sonno di morte.
\end{altverse}


\end{poem}

\begin{poem}{Haiku III}{}

\settowidth{\versewidth}{non meno è ciò che meriti.}

\begin{altverse}
% Incipits are used in the ToP if no title is given.
\incipit{Ruota il cielo.}\\*
Ordigno d'apparenze, \\*
liquida luce.
\end{altverse}


\end{poem}

\begin{poem}{Haiku IV}{}

\settowidth{\versewidth}{non meno è ciò che meriti.}

\begin{altverse}
% Incipits are used in the ToP if no title is given.
\incipit{Suonano i pini.}\\*
La risata di un campo.\\*
È primavera.
\end{altverse}


\end{poem}

\begin{poem}{Haiku V}{}

\settowidth{\versewidth}{non meno è ciò che meriti.}

\begin{altverse}
% Incipits are used in the ToP if no title is given.
\incipit{Bocche di luce.}\\*
Poggia lo stelo curvo\\*
nella mattina.
\end{altverse}


\end{poem}

\begin{poem}{}{}

\settowidth{\versewidth}{non meno è ciò che meriti.}

\begin{altverse}
% Incipits are used in the ToP if no title is given.
\incipit{Negli anni mi accompagna felice}\\*
il tuo ricordo. Travagliato\\*
dalle inclemenze del tempo,\\*
quello è una vecchia casa\\*
nella parte più antica del mio cuore.\\*
Due edere, gemelle,\\*
si contendono il dubbio privilego\\*
della sua facciata cadente.\\*
Amo quella casa, dove nell'inedia\\*
dei miei afosi quindic'anni\\*
ho imparato l'essenziale.\\*
Lì ho avuto il debutto malcerto\\*
da liceale insipiente e selvatico,\\*
l'iniziazione a misteri antichi.\\*
Ancora non conoscevo le donne:\\*
non sapevo che hanno un buon odore;\\*
né che il grano matura al sole\\*
e solo a luglio lo si coglie\\*
- così Proserpina vuole -.
\end{altverse}


\end{poem}


\begin{poem}{}{}

\settowidth{\versewidth}{non meno è ciò che meriti.}

\begin{altverse}
% Incipits are used in the ToP if no title is given.
\incipit{Cade sul pavimento la chiave}.\\*
La porta sbattuta,\\*
violentemente si chiude,\\*
mentre le urla\\*
ancora non si sono spente.
\end{altverse}

\begin{altverse}
A ritroso seguo un filo rosso:\\*
rumore di parole, \\*
qualcosa che muore,\\*
due persone divise da un tavolo\\*
nella stanza piena di luce.
\end{altverse}

\begin{altverse}
A ritroso seguo il filo che\\*
arrotolandosi nelle mie mani\\*
si fa gomitolo, si fa lana\\*
per sparire, poi, nell'inizio\\*
di tutte le cose.
\end{altverse}

\end{poem}

\begin{poem}{}{}

\begin{altverse}
% Incipits are used in the ToP if no title is given.
\incipit{Non andartene, docile compagna}.\\*
Nella casa si allungano le ombre,\\*
ma una candela arde nella cucina.\\*
Non andare per la strada solitaria,\\*
viandante a cui nessuno bada, \\*
indifferente agli sguardi.\\*
Resta con me, \\*
perchè la sera è buia\\*
e il giorno è già al tramonto.\\*
Resta con me\\*
perchè la notte mi fa paura\\*
e solo una candela arde fievole nella cucina.
\end{altverse}

\end{poem}

\end{document}